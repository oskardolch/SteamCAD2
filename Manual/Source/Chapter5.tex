\chapter{Rasters}\label{chap:chap5}
\special{pdf: out 1 << /Title (\thechapter. Rasters) /Dest [ @thispage /FitH @ypos ] >>}

SteamCAD2 also allows a raster image to be inserted and registered. Registering a raster
means to draw three lines, selecting one point wherever on the image and connecting it
to a point in your vector drawing space. Three points allow an affine transformation
to be applied on the raster image and fit your vector drawing.

\section{Importing rasters}\label{sec:impraster}
\special{pdf: out 2 << /Title (\thesection. Importing rasters) /Dest [ @thispage /FitH @ypos ] >>}

Selecting this menu will open a file chooser dialog, with default filter for selecting
images of type PNG, JPEG, GIF and TIFF. Other images, such as Windows bitmaps may also
be loaded, but there is a risk of the saved file not to be portable to other operating
systems.

Also note that the memory handling of the raster images is not any sophisticated, so don't
attempt to load a big files. Use a 3rd party image manipulation programs, such as GIMP, to
reduce the image size and possibly to convert it to a suitable format.

Once an image is imported, it can be selected (by its border) and manipulated as most
of other objects - can be moved, rotated and mirrored. However, the most useful
operation is probably registering the image. This is a process when the image gets
properly alligned with other drawing objects.

\section{Registering rasters}\label{sec:regraster}
\special{pdf: out 2 << /Title (\thesection. Registering rasters) /Dest [ @thispage /FitH @ypos ] >>}

To register a raster, you need to select one raster image in your drawing. The raster
images can only be selected by its border. Although it reminds an area object in some
aspects, it is not possible to select it by clicking inside the image area. The reason
is simple - you most likely need to insert an image as a background for your drawing.
So it is necessary to draw over the image as if there is no image. If the image is
selectable by the interior, all the snaping and selecting objects above the image would
be very inconvenient.

Once you enter the menu command, you will be prompted to draw three lines, which connect
three points on the image (can actually be also outside the image, but it would most
likely have no meaning) with three points in your drawing. The points on the image
must not lie in one line, otherwise the transformation matrix gets singular and cannot
be inverted.

After you finish the third line, the image gets transformed by an affine transformation
so that it fits the three points on the drawing best.

\section{Show/Hide rasters}\label{sec:showraster}
\special{pdf: out 2 << /Title (\thesection. Show/Hide rasters) /Dest [ @thispage /FitH @ypos ] >>}

Sometimes it is good to visually verify the drawing so far done on the image. For this
reason there are two commands available - Hide raster (Ctrl + H) and Show raster (Alt + R).

\section{Samples}\label{sec:samples}
\special{pdf: out 2 << /Title (\thesection. Samples) /Dest [ @thispage /FitH @ypos ] >>}

Few more samples are included in SteamCAD2. Let us explain them a bit.

\subsection{264\_0.sc2}
This is a drawing of Austrian steam engine 6 class. After the empire split, they became
the class 264.0 at Czechoslocak railways. The scale of this drawing is 1:87 (H0).

\subsection{SR\_class\_S15.sc2}
This is a drawing of British Southern Railway slass S15. This sample demonstrates usage
of imperial units. The scale is 1:76.

\subsection{374\_0\_xxx.sc2}
This is a draing of Hungarian class 327. The ``base'' version is preserved to show elements
which are then broken on the ``detail'' version. The scale is 1:45 (0).

\subsection{csd\_mark.sc2}
This is a drawing of the Czechoslovak railways used approximately from 1920 to 1939. The
drawing scale is 1:1.

