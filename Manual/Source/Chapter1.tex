\chapter{Drawing Basics}\label{chap:chap1}
\special{pdf: out 1 << /Title (\thechapter. Drawing Basics) /Dest [ @thispage /FitH @ypos ] >>}

There are 8 line types available in SteamCAD2 - line, circle, ellipse, ``arc ellipse'',
which is an ellipse-like shape composed from circular arcs, parabola, hyperbola,
spline and evolventa.

Before we discuss each of the basic elements, we should mention one aspect also
(probably) unique to SteamCAD2. From geometry point of view, the elements can be either
bound or unbound. To draw a bound element, like circle or ellipse, it is quite easy.
However, the question is what to do with unbound elements, like line, parabola or
hyperbola? In SteamCAD2, when you draw an unbound element, it exists in the drawing
as a ``whole''. What does it mean? It means that if you draw for example a hyperbola,
and then zoom out, the hyberbola ends at the drawing view. You can zoom out as
much as you want and the whole hyperbola is still there. And it remains like that
until you cut it from two sides. Then it becomes a bounded object.

Each drawing primitive has one or more modes how to be drawn. The modes comprise of
mouse button and key combinations. However, in most situation, holding the Ctrl key
while placing a point will snap it to the invisible grid. The grid size can be set
in document properties. Some curved primitives also require one or two line primitive
to be selected prior placing the curve. We will explain all the modes in this chapter.

\section{Line}\label{draw:line}
\special{pdf: out 2 << /Title (\thesection. Line) /Dest [ @thispage /FitH @ypos ] >>}

Line can be started by pressing ``L'' shortcut. Once you are in line mode, you can place
the first point by clicking the left mouse button. When the first point is set, the
behavior of the Ctrl control changes. For placing the second point of a line with Ctrl,
the point is snapped so that the line angle fits certain angles. The ``angular grid''
is set to 15 degrees by default, but can be set in the document properties.

Line has a second placing mode, which is if you start the first point by right mouse
button click. In this case, the line is drawn so that it mirrors the second point into
the first point. This is useful if you need to find a middle point of two points.

When entering the line mode, there is also an edit box on the status bar. You can
specify the angle the line should incline from the horizontal line.

\section{Rectangle}\label{draw:rect}
\special{pdf: out 2 << /Title (\thesection. Rectangle) /Dest [ @thispage /FitH @ypos ] >>}

Rectangle is a convenience shortcut for creating a rectangular path. It can be invoked
by pressing ``R'' shortcut. When entering the first point, width and/or height can be
fixed by entering a value into corresponding edit boxes in the status bar.

\section{Circle}
\special{pdf: out 2 << /Title (\thesection. Circle) /Dest [ @thispage /FitH @ypos ] >>}

The circle can be started by pressing ``C'' shortcut and can operate in four modes.
\begin{enumerate}
\item Plcaing the centre by right mouse button and a point on the circle by left
mouse button.
\item Placing three points on the circle by left mouse button clicks.
\item Placing two poins defining the diameter by first left mouse button click and
left mouse button double click.
\item Placing two points on the circle by left mouse button clicks while keeping
the circle cetner on a selected line.
\end{enumerate}

When placing a circle, there is also an edit box on the status bar, wehere you can
specify the circle radius.

\section{Ellipse}
\special{pdf: out 2 << /Title (\thesection. Ellipse) /Dest [ @thispage /FitH @ypos ] >>}

The ellipse can be started by pressing ``E'' shortcut and can operate in three modes.
\begin{enumerate}
\item Placing two focus points by right mouse button click and one point on the ellipse
by left mouse button.
\item When two non-parallel lines are selected, left mouse double click places the ellipse
so that the ellipse center lies on the lines intersection and the major and minor ellipse
semiaxes mirros the selected lines.
\item When two non-parallel lines are selected, two left mouse clicks define two points
on the ellipse while the ellipse center lies at the two selected lines intersection.
\end{enumerate}

When placing with selected intersecting ``construction'' lines, the ellipse is constructed
so that the tanget on the ellipse and a construction line intersection is parallel to the
second construction line. This can be useful when drawing in axonometric projection.

\section{Arc ellipse}
\special{pdf: out 2 << /Title (\thesection. Arc ellipse) /Dest [ @thispage /FitH @ypos ] >>}

Arc ellipse is a special primitive, which in fact could be replaced by four circular arcs.
However, it is there to help mimic the old style of hand drawing, when creating an exact
ellipse was difficult, and the ellipse was often approximated by a shape consisting of
circular arcs.

It has two modes of operation. Two non-parallel line primitives must always be selected,
and the arc ellipse can either be placed by one left mouse button double click, or by
two consecutive left mouse button clicks. The shortcut for arc ellipse is ``A''.

\section{Hyperbola}
\special{pdf: out 2 << /Title (\thesection. Hyperbola) /Dest [ @thispage /FitH @ypos ] >>}

To place a hyperbola, two non-parallel lines must be selected. The hyperbola is then
placed by left mouse button click and this is the only mode. The shortcut for hyperbola
is ``H''.

\section{Parabola}
\special{pdf: out 2 << /Title (\thesection. Parabola) /Dest [ @thispage /FitH @ypos ] >>}

Parabola always requires one line primitive to be selected. It is then placed by single
left mouse button click and the clicked point represents the parabola's focus. Parabola
can be invoked by ``P'' shortcut.

\section{Spline}
\special{pdf: out 2 << /Title (\thesection. Spline) /Dest [ @thispage /FitH @ypos ] >>}

Spline places a second order B-spline curve. The curve is designed so that it goes through
the clicked points. The points are placed by left mouse button click and the number of
points is not limited. Clicking the right mouse whenever during the curve creation toggles
the closeness. Initially, the spline is open. When clicking the right mouse button, the spline
becomes a closed curve. Spline can be invoked by pressing ``S'' shortcut.

Double click the left mouse button to finish the curve. The double clicked point becomes
part of the curve definition. Sometimes it may happen that you make single click instead
of double. In this case, you can still save the spline with the last placed point by
pressing the ``Enter'' key. You can also remove last point from the spline by pressing
the ``Delete'' button.

In SteamCAD2 you can also extend already finished saved spline using the Extend command,
this will be discussed more in the section~\ref{sec:extend}.

\section{Evolventa}
\special{pdf: out 2 << /Title (\thesection. Evolventa) /Dest [ @thispage /FitH @ypos ] >>}

This is a curve created by rolling out a line along a circle. This is included in SteamCAD2
as a gear creation helper. To start evolventa, one circle must be selected. Start evolventa
by pressing ``V'' shortcut and place two points by left mouse button click to define two
points on the evolventa.

\section{Placing a line tangent to two curves}
\special{pdf: out 2 << /Title (\thesection. Placing a line tangent to two curves) /Dest [ @thispage /FitH @ypos ] >>}

Line is a special curve and has more placing options than the two mentioned in the
section~\ref{draw:line}. We have already mentioned that holding Ctrl key while positioning
the second point keeps the point restricted at certain angles. Moreover, if one primitive
(of arbitrary type) is selected while dwawing a line, the angle restricted for Ctrl key
is derived from the selected curve.

This can be used to draw a line perpendicular to another one or a line tangent to a curve.
If a primitive is selected and you put the first line point on that curve, holding the Ctrl
key will restrict to angle of the line with respect to the tangent of the selected curve.

A special and quite frequent task in mechanical engineering is to draw a line which is tangent
to two selected curves, usually circles. SteamCAD2 can do this, even not restricting the curve
type to circle. The procedure can be as folows:
\begin{enumerate}
\item Select one curve you want to place the line as a tangent.
\item Switch to line mode, and put the first point somewhere on the selected curve.
\item Hold down Ctrl key and move the line so that it gets the required angle. In this case
we want the line position to be tangent, but can be any angle available through the angular
grid, so the line can also be perpendicular, for example.
\item Once you have the desired angle, press Shift key (while still holding the Ctrl key)
and move the line towards the second curve. The line will start following the first curve
while keeping the initial angle.
\item When your mouse pointer gets close to the second curve, use the snapping mechanism
to find either the tangent or the perpendicular point.
\item Click left mouse button to finish the line.
\end{enumerate}

\section{Snapping}
\special{pdf: out 2 << /Title (\thesection. Snapping) /Dest [ @thispage /FitH @ypos ] >>}

When designing SteamCAD2, the first idea was to put various options for snapping as we can
see in other CAD systems, like snap on lines, intersections, line ends etc. In the end
I decided that SteamCAD2 has a fixed snap logic. It snappes to curves, curve ends,
intersections, tangent points and perpendicular points. The snap to pure curve element has
approximatelly half gravity than the other elements.

Considering that an arbitrary zoom is available in SteamCAD2, it gives you an oportunity
to snap at the intended point in most situations. When it happens that the snap is
impossible - usually due to too many elements close each other, one can right click the
unwanted element and select ``Disable snap'' from the pop-up menu. It is also possible to
disable the snap for the whole selection, using the ``Alt+N'' shortcut.

The disabled snapping can be later restored either by right click or by ``Alt+N'' command.
And the snapping property is not persistent in the model, so saving the document and reopen
it will clear all the dissabled snaps.

\section{View manipulation}
\special{pdf: out 2 << /Title (\thesection. View manipulation) /Dest [ @thispage /FitH @ypos ] >>}

As a final topic of this chapter, let's look at the view manipulation. There are two commands
for manipulating the view available in the menu - it is the \textbf{Best Fit} and
\textbf{Normal Size}. The best fit in most drawing applications calculates the minimal bounding
box of all objects in the view and zooms to this. Since SteamCAD2 can contain unbounded objects,
there is no meaning for such a command. So the best fit in SteamCAD2 simple zooms to the whole
drawing area (representing the defined paper) with some margin.

The normal size command tries to zoom the view so that it's physical units correspond to the
dimensions on the screen, using the available monitor properties. It works with resonable
accuracy on most of the physical monitors, and with less accuracy in virtualized environments.

The other two common view opeartions - \textbf{zoom} and \textbf{pan} - can only be managed
by the mouse. This is why a 3-button mouse with wheel is necessary. Zoom is simply performed
by rotating the while, while panning is performed by clicking the wheel and dragging the view.

These operations are used so heavily in technical drawing so that it does not make sense
to create a menu item for them. Moreover, these zoom and pan actions are available always,
it means regardless the drawing mode you are currently in, you can still use the wheel
to zoom and pan. The wheel is not used by any other command so it is not in a conflict
with anything else, thus always ready to manipulate the view.

Finally there two more commands to show the snapping grid. Those are \textbf{Show Grid Points}
and \textbf{Show Grid Lines}. The meaning is obvious - the grid lines and points can be toggled
on or off, and they both can be combined together.
